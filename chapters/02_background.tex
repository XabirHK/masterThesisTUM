% !TeX root = ../main.tex
% Add the above to each chapter to make compiling the PDF easier in some editors.

\chapter{Background}\label{chapter:background}

\section{Sensor}

A sensor is a device that can  monitor the environment that its designed for. Senson can be a single pice of hardware or can be a combination of hardware  accuaring multiple information.
Sensors vary in quality and price. Many modern sensors have on-board
processing units to understand the data acquired from the sensing element without
a separate computational platform. Remote sensing devices are sensors that can
perceive information without physical contact. Unfortunately, sensors tend to have
inherent problems including, but not limited to [Elmenreich, 2001]:
\begin{itemize}
\item Spatial coverage limits: Each sensor may only cover a certain region of space. For example, a dashboard camera will observe less surrounding region than a
camera with a wide-view lens.
\item Temporal coverage limits: Each sensor may only provide updates over certain periods of time, which may cause uncertainty between updates.
\item  Imprecision: Each sensor has limits to its sensing element. • Uncertainty: Unlike imprecision, uncertainty varies with the object being observed rather than the device making the observation. Uncertainty may be
introduced by many environment factors or sensor defects in addition to time
delay.

\item Deprivation of sensor: Sensor element breakdown will cause loss of perception in environment.
Many different types of sensors exist in the world, and each has its own unique ap
plication. Four sensors are pertinent to the field of autonomous driving [Levinson et al., 2011].
The four sensors, their descriptions, uses, advantages and disadvantages are men
tioned below:
\item GPS: Global-positioning system (GPS) is a system of satellites and receivers used for global navigation of Earth designed by the U.S. military. GPS sends a
signal to any GPS receiver with an unobstructed line of sight to four or more
GPS satellites surrounding Earth [gps, 2011]. GPS is useful for finding the
exact coordinates of a vehicle when it is in the line of sight of multiple satellites
orbiting the Earth.

– Advantages: Precise coordinate measurements, fast, reliable in line of
sight, externally-managed satellite systems.
– Disadvantages: Expensive, subject to failure in bad weather conditions,
subject to failure in distant locations where satellite coverage is blocked or
unavailable, dependent on external data source, subject to hijacking and
interference.

\item Radar: RAdio Detection And Ranging (radar) is a remote sensing device that uses an antenna to scatter radio signals across a region in the direction it is
pointing and listens for response signals that are reflected by objects in that area.
Radar measures signal time of flight to determine the distance. Radars may use
the doppler effect to compute speed based on shift in frequency of scattered
waves as an object moves. Radar is useful for detecting obstacles, vehicles and
pedestrians around a vehicle [Huang et al., 2016]. Tracking multiple targets at
once is a primary use for an automotive radar.
– Advantages: High-bandwidth signals, wide-spread area coverage, inde
pendent from external systems, works in multiple weather conditions, light
independent solution.
– Disadvantages: Expensive, subject to interference, easy to corrupt signal
with electromagnetic interference, many reflective radio responses make it
harder to manage radar signals, algorithms for radar tracking are still
imperfect, narrow field-of-view.
\item Camera: A camera is an optical instrument that utilizes at least one converging or convex lens and a shutter to limit light intake into an enclosed housing for
capturing images or recording image sequences [Kodak, 2017]. Video cameras
work much like still-image cameras, but instead of simply capturing still images,
they record a series of successive still images rapidly at a specific frame rate

[Kodak, 2017]. A camera is useful for acquiring images or video sequences of
object pixels in view of the lens in order to help detect, segment, and classify
objects based on perceivable object properties like location, color, shape, edges
and corners.
– Advantages: Perceives high-level object characteristics like color, shape,
and edges, perceives location relative to camera unit, easy to visualize
data.
– Disadvantages: Potential slow frame-rate update, image quality may
be dependent on light, weather and various other factors, data-intensive
processing, all cameras perceive objects differently, typically has a limited
range of perception compared to other sensors, may be expensive.
 \item LiDAR: Light Detection and Ranging (LiDAR) is a method of remote sensing that uses light in the form of a pulsed laser to measure distance to an object
based on signal time of flight [NOAA, 2012]. LiDAR is useful for perceiving
surroundings when 3-dimensional, high-resolution, light-independent images are
necessary.
– Advantages: Independent of light, weather and external data sources,
fast, accurate, 3-dimensional, high-resolution.
– Disadvantages: Expensive, subject to interference by reflection or lack
thereof, incompatible with transparent surfaces, data-intensive processing,
less durable than other sensors.

\end{itemize}

\section{Sensor Fusion}
Sensor fusion is the act of combining data acquired from two or more sensors sources
such that the resulting combination of sensory information provides a more certain description of factors observed by the separate sensors than would be if used individ
ually [Elmenreich, 2001]. Sensor fusion is pertinent in many applications that entail
the use of multiple sensors for inference and control. Examples of applications in
clude intelligent and automated systems such as automotive driver assistance systems,
autonomous robotics, and manufacturing robotics [Elmenreich, 2007].
Sensor fusion methods aim to solve many of the problems inherently present in
sensors. Several important benefits may be derived from sensor fusion systems over
single or disparate sensor sources. The benefits of sensor fusion over single source are
the following [Elmenreich, 2001]:
\begin{itemize}
 \item Reliability: Using multiple sensor sources introduces more resilience to partial sensor failure, which leads to greater redundancy and reliability.
 \item Extended spatial coverage: Each sensor may cover different areas. Combining the covered areas will lead to a greater overall coverage of surrounding
environment and accommodate sensor deprivation.
 \item Extended temporal coverage: Each sensor may update at different time intervals, and thus interpolated sensor updates can be joined for increased tem
poral coverage and decreased sensor deprivation.
 \item Increased Confidence: Combining sensor data will provide increased confidence by providing measurements resilient to the uncertainties in any particular
sensor based on the combined coverage and error mitigation of all sensors. \item  Reduced Uncertainty: Given the resilience of multiple sensors to the specific uncertainty of any one, the overall uncertainty of the perception system can be
drastically reduced using sensor fusion.
 \item Robustness against noise: Multiple sensor sources can be used to determine when any one sensor has encountered noise in order to mitigate influence of
noise in the system.
 \item Increased Resolution: Multiple sensor sources can be used to increase the resolution of measurements by combining all observations.
\end{itemize}
According to [Rao, 2001], sensor fusion can yield results that outperform the measurements of the single best sensor in the system if the fusion function class satisfies a
proposed isolation property based on independently and identically distributed (iid)
samples. The fusion function classes outlined that satisfy this isolation property
are linear combinations, certain potential functions and feed-forward piecewise linear
neural networks based on minimization of empirical error per sample.



\section{Current Sensor Fusion Systems}
Many researchers and companies have developed their own versions of sensor fusion
systems for various purposes. Many of these systems are well-known an widely used
in practice within the fields of automotive and robotics. In [Steux et al., 2002], a
vehicle detection and tracking system using monocular color vision and radar data
fusion using a 3-layer belief network was proposed called FADE. The fusion system
focused on lower-level fusion and combined 12 different features to generate target
position proposals each step and for each target. FADE performed in real-time and
yielded good detection results in most cases according to scenarios recorded in a real
car.
A fusion system for collision warning using a single camera and radar was applied
to detect and track vehicles in [Srinivasa et al., 2003]. The detections were fused using
a probabilistic framework in order to compute reliable vehicle depth and azimuth
angles. Their system clustered object detections into meta-tracks for each object and
fused object tracks between the sensors. They found that the radar had many false
positives due to multiple detections on large vehicles, structures, roadway signs and overhead structures. They also found that the camera had false positive detections on
larger vehicles and roadway noise like potholes. Their system worked appropriately
for nearby vehicles that were clearly visible by both sensors, but the system failed to
detect vehicles more than 100 meters away due to insufficient resolution or vehicle
occlusion.
In [Dagan et al., 2004], engineers from Mobileye successfully applied a camera
system to compute the time to collision (TTC) course from size and position of
vehicles in the image. Although they did not test this theory, they mentioned the
future use of radar and camera in a sensor fusion system since the radar would
give more accurate range and range-rate measurements while the vision would solve
angular accuracy problems of the radar. When the research was conducted, it was
suggested that the fusion solution between radar and camera was costly, but since
then, costs have decreased.
A collision mitigation fusion system using a laser-scanner and stereo-vision was
constructed and tested in [Labayrade et al., 2005]. The combination of the compli
mentary laser scanner and stereo-vision sensors provided a high detection rate, low
false alarm rate, and a system reactive to many obstacle occurrences. They men
tioned that the laser-scanner was fast and accurate but could not be used alone due
to many false alarms from collisions with the road surface and false detections with
laser passes over obstacles. They also mentioned that stereo-vision was useful for
modeling road geometry and obstacle detection, but it was not accurate for comput
ing precise velocities or TTC for collision mitigation.
In [Laneurit et al., 2003], a Kalman filter was successfully developed and applied
for the purpose of sensor fusion between multiple sensors including GPS, wheel angle
sensor, camera and LiDAR. They showed that this system was useful for detection
and localization of vehicles on the road, especially when using the wheel angle sensor
for detecting changes in vehicle direction. Their results revealed that cooperation between the positioning sensors for obstacle detection and location paired with LiDAR
were able to improve global positioning of vehicles.
A deep learning framework for signal estimation and classification applicable for
mobile devices was created and tested in [Yao et al., 2016]. This framework applied
convolutional and recurrent layers for regression and classification mobile comput
ing tasks. The framework exploited local interactions of different sensing modalities
using convolutional neural network (CNN)s, merged them into a global interaction
and extracted temporal relationships via stacked GRU or LSTM layers. Their frame
work achieved a notable mean absolute error on vehicle tracking regression tasks as
compared to existing sensor fusion systems and high accuracy on human activity
recognition classification tasks while it remained efficient enough to use on mobile
devices like the Google Nexus 5 and Intel Edison.
A multimodal, multi-stream deep learning framework designed to tackle the ego
centric activity recognition using data fusion was proposed in [Song et al., 2016b]. To
begin, they extended a multi-stream CNN to learn spatial and temporal features from
egocentric videos. Then, they proposed a multi-stream LSTM architecture to learn
features from multiple sensor streams including accelerometer and gyroscope. Third,
they proposed a two-level fusion technique using SoftMax classification layers and dif
ferent pooling methods to fuse the results of the neural networks in order to classify
egocentric activities. The system performed worse than a hand-crafted multi-modal
Fisher vector, but it was noted that hand-crafted features tended to perform better
on smaller datasets. In review of the research, it seems there were limited amounts of
data, flaws in the fusion design with SoftMax combination and flaws in the sensors,
such as limited sensing capabilities. These factors all may have led to worse results
than hand-crafted features on the utilized dataset.
In [Wu et al., 2015], a multi-stream deep fusion neural network system using con
volution neural networks and LSTM layers was applied to classify multi-modal temporal stream information in videos. Their adaptive multi-stream fusion system achieved
an accuracy level much higher than other methods of fusion including averaging, ker
nel averaging, multiple kernel learning (MKL), and logistic regression fusion methods.



\section{Machine Learning}
Machine Learning is a term that describes algorithms that learn from data.
Goodfellow et al. quotes Mitchell from 1996 for a definition of
what learning means, A computer program is said to learn from experience
E with respect to some class of tasks T and performance measure P, if its
performance at tasks in T, as measured by P, improves with experience E.
There is a wide variety of tasks T that can be done. For instance common
tasks are regression, classification, anomaly detection, denoising and translation.
Depending on what experience E they see during the learning process the most
algorithms can be classified as supervised or unsupervised learning. Supervised
learning algorithms experience a data set together with target values or labels
that act as instructions on what to do. The unsupervised learning algorithms
don’t have any labels or targets but instead try to learn about the structure of
the data set. The performance P can be how far off from the labels the output
of the algorithm is in the case of supervised learning. This performance measure
is often difficult to choose.



\section{Deep Learning}
Deep Learning is a term used for training deep artificial neural networks, or
ANN s. An ANN is a function approximator consisting of multiple functions,

also called neurons. An ANN is called a feedforward ANN if there are no
feedback connections of the output of the ANN being sent to itself. Each neuron
is a linear function of some inputs, fed into a nonlinear function, called activation
function. If the input of a neuron is a vector x, the output of the neuron gi can
be written as


\subsection{Convulitional Neural Network}
The non-linear functions called activation functions, such as K in Equation 2.1,
are typically fixed non-linear functions that are used to make the ANN able to
approximate non-linear functions. If the non-linear activation functions weren’t
used, the output of the ANN would still be a linear function of the inputs x. In deep learning, there are a few commonly used functions that
have become standard as activation functions.
The rectifying linear unit, or ReLU is a function that is defined as


\subsection{Back-propagation}
When training a ANN to approximate a function, gradient based optimization
is commonly used. To compute the gradient of a (loss) function f with
respect to the parameters  an algorithm called back-propagation is commonly
used. It back-propagates from the objective function to gradients of the different
weights and biases in the ANN to compute the gradient of the objective function with respect to all the parameters. If we have a function 
can be a cost function in a supervised learning setting but could also be other
functions like a reward function in the reinforcement learning setting that will

\subsection{Gradient based optimization for deep learning}

The ANN weights and biases are updated to optimize some objective function
with gradient based optimization, using the gradients computed with the backprop algorithm.
The classic algorithm for optimizing the parameters in a ANN is called Gradient
Descent. be an objective function that we want to minimize, where
x is the input to the ANN the trainable parameters of the ANN. Then
moving in the opposite direction of the gradient of L with respect to  we move
the parameters in a direction that makes the objective function smaller, which is
what we want. However, usually x are random samples and the true gradient is
the expected value of the gradient with respect to the random samples actually
used. Thus when computing the gradient of the loss function as a function of
some samples x, we are computing an unbiased noisy estimate of the gradient.
This is referred to as Stochastic Gradient Descent, or SGD. Stochastic Gradient

\subsection{Batch Normalization}
Batch Normalization is a recent method in deep learning used to be able to train
networks faster and use higher learning rates with decreased risk of divergence.
It does so by making the normalization a part of the model architecture, fixing
the mean and variances of the inputs to a layer by a normalization step. This
makes the risk of the inputs to the activation functions getting in a range where
the gradient vanishes smaller and allows for the use of higher learning rates. 

\section{Deep Learning}
Deep Learning is a term used for training deep artificial neural networks, or
ANN s. An ANN is a function approximator consisting of multiple functions,

also called neurons. An ANN is called a feedforward ANN if there are no
feedback connections of the output of the ANN being sent to itself. Each neuron
is a linear function of some inputs, fed into a nonlinear function, called activation
function. If the input of a neuron is a vector x, the output of the neuron gi can
be written as


\subsection{Activation Functions}
The non-linear functions called activation functions, such as K in Equation 2.1,
are typically fixed non-linear functions that are used to make the ANN able to
approximate non-linear functions. If the non-linear activation functions weren’t
used, the output of the ANN would still be a linear function of the inputs x. In deep learning, there are a few commonly used functions that
have become standard as activation functions.
The rectifying linear unit, or ReLU is a function that is defined as

\subsection{Cost Functions}
To train a ANN, cost functions are usually minimized. In supervised learning, where we have labeled data to learn from, the cost functions are more
straightforward than for instance in reinforcement learning that we will discuss
in section 
The two main problems of supervised learning are regression and classification.
In regression one wants to predict a numerical value whereas in classification the
goal is to predict which class something belongs to given some inputs. Different
loss functions are common for regression and classification. Most of them are
however derived from the same principle, the one of Maximum Likelihood.

\subsection{Back-propagation}
When training a ANN to approximate a function, gradient based optimization
is commonly used. To compute the gradient of a (loss) function f with
respect to the parameters  an algorithm called back-propagation is commonly
used. It back-propagates from the objective function to gradients of the different
weights and biases in the ANN to compute the gradient of the objective function with respect to all the parameters. If we have a function 
can be a cost function in a supervised learning setting but could also be other
functions like a reward function in the reinforcement learning setting that will

\subsection{Gradient based optimization for deep learning}

The ANN weights and biases are updated to optimize some objective function
with gradient based optimization, using the gradients computed with the backprop algorithm.
The classic algorithm for optimizing the parameters in a ANN is called Gradient
Descent. be an objective function that we want to minimize, where
x is the input to the ANN the trainable parameters of the ANN. Then
moving in the opposite direction of the gradient of L with respect to  we move
the parameters in a direction that makes the objective function smaller, which is
what we want. However, usually x are random samples and the true gradient is
the expected value of the gradient with respect to the random samples actually
used. Thus when computing the gradient of the loss function as a function of
some samples x, we are computing an unbiased noisy estimate of the gradient.
This is referred to as Stochastic Gradient Descent, or SGD. Stochastic Gradient



\section{Sensors}

A sensor is a piece of hardware that monitors an environment based on a sensing
element. Sensors vary in quality and price. Many modern sensors have on-board
processing units to understand the data acquired from the sensing element without
a separate computational platform. Remote sensing devices are sensors that can perceive information without physical contact. Unfortunately, sensors tend to have
inherent problems including, but not limited to [Elmenreich, 2001]:
• Spatial coverage limits: Each sensor may only cover a certain region of space. For example, a dashboard camera will observe less surrounding region than a
camera with a wide-view lens.
• Temporal coverage limits: Each sensor may only provide updates over certain periods of time, which may cause uncertainty between updates.
• Imprecision: Each sensor has limits to its sensing element. • Uncertainty: Unlike imprecision, uncertainty varies with the object being observed rather than the device making the observation. Uncertainty may be
introduced by many environment factors or sensor defects in addition to time
delay.
• Deprivation of sensor: Sensor element breakdown will cause loss of perception in environment.
Many different types of sensors exist in the world, and each has its own unique ap
plication. Four sensors are pertinent to the field of autonomous driving [Levinson et al., 2011].
The four sensors, their descriptions, uses, advantages and disadvantages are men
tioned below:
• GPS: Global-positioning system (GPS) is a system of satellites and receivers used for global navigation of Earth designed by the U.S. military. GPS sends a
signal to any GPS receiver with an unobstructed line of sight to four or more
GPS satellites surrounding Earth [gps, 2011]. GPS is useful for finding the
exact coordinates of a vehicle when it is in the line of sight of multiple satellites
orbiting the Earth.

– Advantages: Precise coordinate measurements, fast, reliable in line of
sight, externally-managed satellite systems.
– Disadvantages: Expensive, subject to failure in bad weather conditions,
subject to failure in distant locations where satellite coverage is blocked or
unavailable, dependent on external data source, subject to hijacking and
interference.
• Radar: RAdio Detection And Ranging (radar) is a remote sensing device that uses an antenna to scatter radio signals across a region in the direction it is
pointing and listens for response signals that are reflected by objects in that area.
Radar measures signal time of flight to determine the distance. Radars may use
the doppler effect to compute speed based on shift in frequency of scattered
waves as an object moves. Radar is useful for detecting obstacles, vehicles and
pedestrians around a vehicle [Huang et al., 2016]. Tracking multiple targets at
once is a primary use for an automotive radar.
– Advantages: High-bandwidth signals, wide-spread area coverage, inde
pendent from external systems, works in multiple weather conditions, light
independent solution.
– Disadvantages: Expensive, subject to interference, easy to corrupt signal
with electromagnetic interference, many reflective radio responses make it
harder to manage radar signals, algorithms for radar tracking are still
imperfect, narrow field-of-view.
• Camera: A camera is an optical instrument that utilizes at least one converging or convex lens and a shutter to limit light intake into an enclosed housing for
capturing images or recording image sequences [Kodak, 2017]. Video cameras
work much like still-image cameras, but instead of simply capturing still images,
they record a series of successive still images rapidly at a specific frame rate

[Kodak, 2017]. A camera is useful for acquiring images or video sequences of
object pixels in view of the lens in order to help detect, segment, and classify
objects based on perceivable object properties like location, color, shape, edges
and corners.
– Advantages: Perceives high-level object characteristics like color, shape,
and edges, perceives location relative to camera unit, easy to visualize
data.
– Disadvantages: Potential slow frame-rate update, image quality may
be dependent on light, weather and various other factors, data-intensive
processing, all cameras perceive objects differently, typically has a limited
range of perception compared to other sensors, may be expensive.
• LiDAR: Light Detection and Ranging (LiDAR) is a method of remote sensing that uses light in the form of a pulsed laser to measure distance to an object
based on signal time of flight [NOAA, 2012]. LiDAR is useful for perceiving
surroundings when 3-dimensional, high-resolution, light-independent images are
necessary.
– Advantages: Independent of light, weather and external data sources,
fast, accurate, 3-dimensional, high-resolution.
– Disadvantages: Expensive, subject to interference by reflection or lack
thereof, incompatible with transparent surfaces, data-intensive processing,
less durable than other sensors.

\section{Inertial Mesurement Unit (IMU)}
An inertial measurement unit is an electronic device that measures and reports a body's specific force, angular rate, and sometimes the orientation of the body, using a combination of accelerometers, gyroscopes, and sometimes magnetometers.

\section{Sensorfutions}
Sensor fusion is the act of combining data acquired from two or more sensors sources
such that the resulting combination of sensory information provides a more certain description of factors observed by the separate sensors than would be if used individ
ually [Elmenreich, 2001]. Sensor fusion is pertinent in many applications that entail
the use of multiple sensors for inference and control. Examples of applications in
clude intelligent and automated systems such as automotive driver assistance systems,
autonomous robotics, and manufacturing robotics [Elmenreich, 2007].
Sensor fusion methods aim to solve many of the problems inherently present in
sensors. Several important benefits may be derived from sensor fusion systems over
single or disparate sensor sources. The benefits of sensor fusion over single source are
the following [Elmenreich, 2001]:

• Reliability: Using multiple sensor sources introduces more resilience to partial sensor failure, which leads to greater redundancy and reliability.
• Extended spatial coverage: Each sensor may cover different areas. Combining the covered areas will lead to a greater overall coverage of surrounding
environment and accommodate sensor deprivation.
• Extended temporal coverage: Each sensor may update at different time intervals, and thus interpolated sensor updates can be joined for increased tem
poral coverage and decreased sensor deprivation.
• Increased Confidence: Combining sensor data will provide increased confidence by providing measurements resilient to the uncertainties in any particular
sensor based on the combined coverage and error mitigation of all sensors.
• Reduced Uncertainty: Given the resilience of multiple sensors to the specific uncertainty of any one, the overall uncertainty of the perception system can be
drastically reduced using sensor fusion.
• Robustness against noise: Multiple sensor sources can be used to determine when any one sensor has encountered noise in order to mitigate influence of

noise in the system.
• Increased Resolution: Multiple sensor sources can be used to increase the resolution of measurements by combining all observations.
According to [Rao, 2001], sensor fusion can yield results that outperform the mea
surements of the single best sensor in the system if the fusion function class satisfies a
proposed isolation property based on independently and identically distributed (iid)
samples. The fusion function classes outlined that satisfy this isolation property
are linear combinations, certain potential functions and feed-forward piecewise linear
neural networks based on minimization of empirical error per sample.


