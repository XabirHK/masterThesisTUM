% !TeX root = ../main.tex
% Add the above to each chapter to make compiling the PDF easier in some editors.

\chapter{Experiment Results and Comparison}\label{chapter:experiment_results_and_comparison}

In this chapter we analyze the tests, the possible outcomes and the best network configurations for the tests of our experiments. The IMU-2/4/8/16 data sets have been used for our model testing. We think that training with our data sets is enough to allow a choice for a particular system , for example by using particular IMU numbers instead of 2/4/8/16. Of those four datasets, we have trained 128 models and validated the models with invisible test results. The evaluation of these models gave us a good understanding of the efficiency and the usefulness of the model in predicting the object's location.

For the assessment, we decided to concentrate on preparation and loss of validity, which uses the loss functions of MAE and MSE. The prediction accuracy of models is also tracked and measured. The calculation of prediction accuracy makes little sense because so many floating points have to be taken into account when calculating accuracy. The prediction precision (both position and orientation) of displacements gives us a decent value which we can rely upon because of its many floating points. Thus we decided to evaluate our models by measuring deviations from the displacement predictions.

This allows us to easily understand the performance of the model. In addition, after predicting the Euclidian deviation distance of the results can also be calculated. We used boxplots to consider the distribution to represent the error deviations of the predictions. We also agreed to take the outliers away, so we didn't have too many.

Following the model evaluation, we intended to combine our solutions in order to have a higher level overview with the best and worst models. In the following sections we will analyze the results of testing our trained models, broken down by network form and data set (depending on number of IMUs used. The data sets are used to train various network types, e.g., shrinking and expanding, 2 and 3 levels of CNN layey architecture, specific learning rates and optimizer parameters.  

\section (Results for the Shrinking models)
\subsection(Result with multiple channel)
\subsection(Result with multiple CNN Layer)


\section(Results for the Expanding models)
\subsection(Result with multiple channel)
\subsection(Result with multiple CNN Layer)